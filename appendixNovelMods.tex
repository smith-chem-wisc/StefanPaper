\documentclass[]{article}

%opening
\title{Novel Modification Discovery Heuristic}
\author{}

\begin{document}

\maketitle

\begin{abstract}

\end{abstract}

\section{The Heuristic}


We describe the heuristic approach for augmenting the modification database and the final notch list.
After the conclusion of the search, the PSMs within 1\% FDR are selected, and the corresponding mass differences are grouped into peaks, which are categorized into one of the following categories:
\begin{itemize}
\item \textit{No mass error} Exact match (within a certain noise tolerance) between the peptide corresponding to the MS2 fragmentation and the reported precursor monoisotopic mass.
\item \textit{Monoisotopic error} Peaks around values such as 1.003, 2.006, 3.009 (average differences in mass between the monoisotopic peaks and the first few isotope peks). These correspond to having misidentified the monoisotopic peak in the preprocessing deconvolution step. In this case, the identified peptide is still correct. 
\item \textit{Exact PTM or Adduct Masses}e.g. 42.010 for Acetylation or 21.981 for Sodium adduct. In this analysis, there is no conceptual difference between PTMs and adducts: both are observed simply as mass differences between the identified peptide and the precursor mass.  These peaks are further classified into
\begin{itemize}
\item Localizeable - e.g. Methylation. There is evidence for the modification in the MS2 spectrum: the corresponding peaks are shifted by an appropriate amount. 
\item Labile - e.g. Sulfonation. There is no evidence of the modification in the fragmentation spectrum. 
\item Either Localizeable or Labile, e.g. Phosphorylation. Sometimes there is evidence of it in the fragmentation spectrum, and sometimes there is not. 
\end{itemize}
\item Amino acid removals, additions or substitutions.
\item Modification dependent mass shifts - these peaks occur only in presence of certain modifications in the identified peptide. 	E.g. -15.995 in presence of an identified oxidation, or -79.966 in presence on an identified phosphorylation. 
\item Combinations of any of the above, e.g. 1.987 for combination of a monoisotopic error and a deamidation. Some of these combinations occur frequently, and it is crucial to account for them. 
\end{itemize}

An automated script analyzes each identified peak, and provides clues to the nature of the peak. For every peak, a profile is automatically generated.
It includes the total number of unique peptides associated with the mass shift, the fraction of decoys, mass match with any known entry in the Unimod or UniProt database, mass match to an amino acid addition/removal combination, mass match to a combination of higher frequency peaks, fraction of localizable targets, localization residues and/or termini, and presence of any modifications in the matched peptides.
We follow the following procedure to classify mass shift peaks. 
1.	A z score is computed for the fraction of decoys in the peak, comparing with 1\%FDR. If it corresponds to p>0.05, we do not include this peak in the analysis, because there is no sufficient confidence it produces legitimate results.
2.	

An attempted localization of the mass difference is automatically performed on every peptide-spectrum match. We carefully examine the mass differences with a localization fraction greater than 0.2, and attempt to deduce the chemical formula, specificity sites and positions within a peptide.
Once a determination has been made, we add the new modification type to the modifications list.
An example output of this step is:
 

The improvements of using a Comb search instead of an Open search are two-fold. We summarize the differences in Table 2.
	Open Search	Comb Search
Search Time	54.36 hrs	29.66 hrs
 
The improvement in the discernibility of PTMs is apparent when running a comb search on calibrated vs uncalibrated spectra files. Specifically, the discernibility of PTMs with similar mass errors (ones that are only different because of the mass defect).
This was not really possible previously.
In the current run, Sulfation and Phosphorylation are readily distinguishable.
We present the numbers for the example below in the step 5 numerical validation section.
OPEN SEARCH CASE STUDY
Introduction of the comb search instead of an open search is motivated by a careful review of the Unimod database.
Known modifications with mass difference within 200 Daltons have values that are within [-0.1, 0.2] of every integer.
This interval choice stems from the mass defect in the primary isotope of the common elements.
PTM combinations also have this property.
We tested multiple search strategies in a search for one that gives the smallest number of false positives. 
\subsubsection{Known Modification Confirmation Case Study}

Confirmation of the properties of Sodium Adducts, Phosphorylation and Sulfonation.
\subsubsection{Glycans Case Study}
Proteins that contain modifications such as Hexose, HexNAC, and similar ones are interesting.
The uniprot database only lists some possible sites for such modifications, but does not specify the actual type, thus creating ambiguity regarding the type of the modification.
The new modification discovery workflow is well suited to identify such proteins.
Table~\ref{tbl:glyco} lists the automatically identified glyco modification masses and identities, sorted by frequency of occurence. 

\begin{table}[]
\centering
\caption{Glyco}
\label{tbl:glyco}
\begin{tabular}{lll}
MassShift & Count & UnimodID                        \\
\hline
203.079   & 30    & HexNAc                          \\
162.052   & 18    & Hex                             \\
876.322   & 10    & Hex(2)HexNAc(2)dHex(1)          \\
656.228   & 8     & Hex(1)HexNAc(1)NeuAc(1)         \\
365.132   & 8     & Hex(1)HexNAc(1)                 \\
1216.422  & 8     & Hex(5)HexNAc(2)                 \\
730.265   & 6     & Hex(2)HexNAc(2)                 \\
1378.475  & 4     & Hex(6)HexNAc(2)                 \\
947.323   & 4     & Hex(1)HexNAc(1)NeuAc(2)         \\
963.317   & 3     & Hex(1)HexNAc(1)NeuAc(1)NeuGc(1) \\
178.047   & 3     & Galactosyl                      \\
1038.375  & 3     & dHex(1)Hex(3)HexNAc(2)          \\
146.058   & 2     & dHex                            \\
340.1     & 2     & Glucosylgalactosyl              \\
406.158   & 2     & HexNAc(2)                       \\
697.256   & 2     & HexNAc(2)NeuAc(1)               \\
672.223   & 2     & Hex(1)HexNAc(1)NeuGc(1)         \\
860.326   & 2     & Hex(1)HexNAc(2)dHex(2)         
\end{tabular}
\end{table}

 
\subsubsection{Novel Modification Case Study}


\end{document}
